
\chapter{Variational Inference}{\label{sec:variationalMethods}

\section{Introduction}

Need a systematic way to make inferences.

This entails using probabilities to model our knowledge of the world.



\section{Probability}

Probability distributions can be used to represent our knowledge of the world.
For example, the value of a  experimentally obtained variable will in general
fluctuate around its average.  
If different runs of the experiment are independent
then the  distribution of obtained values fully describes the experiment.

What is learned from a given experiment is then characterised with how the probability distributions
that represent our knowledge change.
If a hypothesis, $\H$, is that a set of experimental data, $\vx = \{x_n|n=1,\ldots,N\}$, should conform to a model with a set of parameters, $\vw = \{w_i| i=1,\ldots,I\}$,
then our full knowledge of the system is given by the joint probability distribution
\begin{align}
  P\lr{\vx,\vw,\H}.
  \label{eqn:fullJointDist}
\end{align}
Of greater importance than \eqnref{fullJointDist}, however,
is to determine how our knowledge of the model changes when we collect the experimental data.
This can be found from  \eqnref{fullJointDist} by splitting the joint distribution into its conditional probabilities.
\begin{align}
  P\lr{\vx,\vw,\H} = P\lr{\vw|\vx,\H}P\lr{\vx|\H}
  =  P\lr{\vx|\vw,\H} P\lr{\vw|\H}.
\end{align}
from which it follows that 
\begin{align}
   P\lr{\vw|\vx,\H} =  \frac{P\lr{\vw|\vw,\H} P\lr{\vw|\H}}{P\lr{\vx|\H}}.
  \label{eqn:BayesTheorem}
\end{align}
Equation \Eqnref{BayesTheorem} is Bayes Theorem.
It states that the probability of the model's parameters, {\em given the data},
can be determined from the probability of the data when the parameters are known, and the a  probability of parameters {\em before the data was known}.
It describes exactly the process of inference.

The term $P\lr{\vx|\vw,\H}$ is the likelihood function.
It evaluates the degree to which the model with a given set of parameters agrees with the experimental data.
If it is assumed that every data point is independent, and that each datum should agree with the prediction of the model, $t_n$, 
to within Gaussian noise
then the likelihood function would be,
\begin{align}
  P\lr{\vx|\vw,\H} = \prod_{n=1}^N \sqrt{\frac{\gamma}{2\pi}}e^{-0.5\gamma\lr{x_n-t_n}^2}.
\end{align}
The variable $\gamma$ is the precision - the inverse of the variance - and is one of the set $\{w_i\}$.

The term $P\lr{\vw|\H}$ in \eqnref{BayesTheorem} is independent of the experimental data $\{x_n\}$.  
It  represents our knowledge of the parameters before the experiment was carried out.
It could be that the parameters are already known to great precision - 
in which case the probability distribution would tend towards a delta function.
Alternatively, it could that the a priori knowledge of the precision, say, 
does not expend beyond the requirement that the precision is positive definite.
In this case the prior distribution would be represented by a scale invariant positive definite distribution.
One such example is the Gamma distribution,
\begin{align}
  P(\gamma|s,c) = \frac{1}{\Gamma(s)c}\lr{\frac{x}{s}}^{c-1}\exp\lr{-\frac{x}{s}},
\label{eqn:Gamma}
\end{align}
in the limit such that $sc = 1$ and $c\rightarrow 0$ \cite{MacKay2003}.

The hypothesis, $\H$,  encompasses all of the assumptions that go into the inference.
These include the choice of the model that is fitted to the data, 
the prior probabilities assigned to the model variables and the 
the noise model described by the likelihood function.
These assumptions are inevitable - they reflect our uncertainty 
 prompts the experiment in the first place.
However, 
since many different hypotheses can be dreamed up,
it is important to be able to be able to evaluate how each is supported by the experimental data.
For this, Bayes Theorem can be applied a second time:
the probability of the hypothesis, given the data, is
\begin{align}
P\lr{\H | \vx } = \frac{P\lr{\vx|H}P\lr{\H}}{P\lr{\vx}}.
\label{eqn:BayesHyp}
\end{align}
Since the probability of the data, $P\lr{\vx}$, 
is independent of the hypothesis
it can be eliminated when comparing two hypotheses, $\H_1$ and $\H_2$,
\begin{align}
\frac{P\lr{\H_1 | \vx }}{P\lr{\H_2 | \vx }} = \frac{P\lr{\vx|\H_1}}{P\lr{\vx|\H_1}}\frac{P\lr{\H_2}}{P\lr{\H_2}}.
\label{eqn:ModelCmp}
\end{align}
The second of the ratios on the right-hand-side of \eqnref{ModelCmp}
give an opportunity, if desired, to prefer one model over another irrespectively of any data collected.
The first quotient is determined from the experimental data.
The term $P\lr{\vx|\H}$ is called the evidence and it is the partition function of  \eqnref{BayesHyp}.

A model that is highly constrained will be inflexible in the range of predictions it can make,
whereas a model that has many free parameters will be able to predict a vast number of possible outcomes.
The more constrained model will therefore have a smaller set of likely outcomes,
but each of these will have a much greater probability than the many possible outcomes of the less constrained model.
The right-hand-side of \eqnref{ModelCmp} therefore directly and quantitively embodies Occan's razor,
the rule of thumb that states that `simpler' models should be favoured over more complicated models.
For a more detailed discussion of model comparison and Occan's razor see \cite[Chapter 28]{Mackay2003}.

To evaluate the evidence the numerator in equation \eqnref{BayesTheorem} must be integrated over the entire parameter space,
\begin{align}
  P\lr{\vx|\H} = \int_\vw d\vw P\lr{\vw|\vw,\H} P\lr{\vw|\H}
\end{align}
In general this cannot be done analytically.
However, it is often the case that the probability density tightly peaked about the maximum.
In this case the evidence may be evaluated by approximating the peak with a Gaussian, which can be integrated.
This is the saddle point approximation.
Expanding the logarithm of the  unnormalised probability distribution, $P^\ast$, 
around the maximum, $\vx_0$,
gives
\eq{
  \ln P^\ast = \ln P^\ast(\vx_0) - \frac{1}{2}\lr{\vx-\vx_0}^T \vA\lr{\vx-\vx_0 }
}
where 
\eq{
\vA = A_{ij} = \frac{\d^2}{\d x_i\d x_j} \ln P^\ast(\vx_0)
}
is the Hessian matrix.
The right-hand-side of equation \eqnref{BayesTheorem}  is therefore approximated by the multidimensional Gaussian
\begin{align}
   P\lr{\vw|\vx,\H} = P^\ast(\vx_0) \exp \lr{- \frac{1}{2}\lr{\vx-\vx_0}^T \vA\lr{\vx-\vx_0 }}
\end{align}
for which the normalisation constant, the evidence, is 
\begin{align}
   P^\ast(\vx_0) \sqrt{\frac{\lr{2\pi}^K}{\det \vA}}
\end{align}

\section{Variational Approach}

Variational methods can be used to approximate a probability distribution, $P$, that is impossible to evaluate exactly, 
with a probability distribution that is more maluable.
The approximation is varied so that it matches the original distribution as closely as possible. 

The amount of information that is lost when a distribution $Q$ is used in place of the distribution $P$ is measured by the relative entropy, 
a quantity known as the  Kullback-Leibler divergence.
%The Kullback Leibler divergence gives a measure of the similarity of two distributions,
It is defined,
\begin{align}
\KLD{Q}{P} &= \int_\vH Q(\vH|\H) \log\frac{Q(\vH|\H)}{P(\vH|\vD,\H)} d\vH,
\end{align}
where $P$ and $Q$ are probability distributions that model a hypothesis, $\H$.
$\vH$ is a set of unknown variables that form the model and $\vD$ is as set of known variables.
From Gibbs inequality it follows that 
\begin{align}
\KLD{Q}{P} \ge 0
\end{align}
and equality is only when $P=Q$.
That is, knowledge of the system is always lost when it is approximated.

The Kullback-Leibler divergence will be minimised in two different ways in this thesis.
The first is 
\nlist{
\item
This property makes the Kullback-Leibler divergence a useful function minimise.
Indeed, using $P(\vH,\vD|\H) = P(\vH|\vD,\H) P(\vD|\H)$, 
we may write,
\begin{align}
  \KLD{Q}{P}  % &=\int_\vH Q(\vH) \log\frac{Q(\vH)}{P(\vH|\vD)} d\vH \\
  &= \int_\vH Q(\vH|\H) \log\frac{Q(\vH|\H)}{P(\vH,\vD|\H)} d\vH  +  \log P(\vD|\H) \\
  &= -S_Q - \int_\vH Q(\vH|\H) \log P(\vH,\vD|\H)  d\vH + \log P(\vD|\H)
\end{align}
where $S_Q = - \int_\vH Q(\vH|\H) \log Q(\vH|\H) d\vH$ is the entropy given the hypothesis%
\footnote{
\begin{quote}
Consider, for example, a crystal of Rochelle salt.
For one set of experiments on it, we work with temperature, pressure and volume.
The entropy can therefore be expressed as some function $S_e(T,P)$.
For another set of experiments on the same crystal,
we work with temperature, the component $e_{xy}$ of the strain tensor,
and the component $P_z$ of the electric polarisation;
the entropy as found in these experiments is a function $S_e(T, e_{xy}, P_z)$.
It is clearly meaningless to ask ``What is the entropy of the crystal?''
unless we first specify the set of parameters which define its thermodynamic state.%

One might reply that in each of the experiments cited, 
we have used only part of the degrees of freedom of the system,
and there is a ``true'' entropy which is a function of all these parameters simultaneously.
However we can always introduce as many parameters as we please...
There is no end to this search for the ultimate ``true'' entropy until we have reached the point where we control
the location of each atom independently.
But just at that point the notion of entropy collapses, and we are no longer talking thermodynamics!

From this we see that entropy is an anthropomorphic concept,
not only in the well known statistical sense that it measures the extent of human ignorance as to the microstate.
{\em Even at the purely phenomenological level, entropy is an anthropomorphic concept.}
For it is a property, not of the physical system,
but of the particular experiments that you or I choose to perform on it.

\flushright Edwin T. Jaynes\cite{Jaynes1965}
\end{quote}
}.
Define the  cost function
\begin{align}
  \L =  \int_\vH Q(\vH) \log P(\vH,\vD) ) d\vH +S_Q  
\end{align}
From which it follows that
\begin{align}
  \L &=   \log P(\vD,\H) - \KLD{Q}{P} \\
     &\le \log P(\vD,\H)
\end{align}
The probability of the model is 
\begin{align}
  P(\H, \vD) &=    \frac{P(\vD, \H) P(\H)}{P(\vD)}\\
             &\le  \frac{\L(Q)P(\H)}{P(\vD)}
\end{align}
\item The second is
}




\section{Independent Components of the pulses}

Assume differing bubble sources as independent components in bubble.

\begin{align}
\vx_t = \vA \vs_t
\end{align}
One model for this is a Gaussian
\begin{align}
  P(x_t| \Lambda) = \G(x_t; AS_t, \Lambda)
\end{align}
However,
from the \figref{} it is seen that the Gaussian noise model is not good.
A better alternative is to use Fourier decomposition,

\begin{align}
  \vx_\omega = \vA \vs_\omega
\end{align}
such that
\begin{align}
  P(\vx_\omega| \Lambda_\omega) = \G( \vx_\omega ; AS_\omega, \Lambda_\omega)
\end{align}

\begin{align}
  P(\vH|\vD) = G(\vH)
\end{align}
\begin{align}
\KLD{Q}{P} &= \int_\vH Q(\vH) \log\frac{Q(\vH)}{P(\vH|\vD)} d\vH \\
  &= \int_\vH Q(\vH) \log\frac{Q(\vH)}{P(\vH,\vD)} d\vH  + \int_\vH Q(\vH) \log P(\vD) d\vH\\
  &= \int_\vH Q(\vH) \log Q(\vH) d\vH - \int_\vH Q(\vH) \log Q(\vH) \log P(\vH,\vD) ) d\vH + \log P(\vD)
\end{align}
Bring cost function
\begin{align}
  \L =  \int_\vH Q(\vH) \log Q(\vH) \log P(\vH,\vD) ) d\vH - Q(\vH) \log Q(\vH) d\vH 
\end{align}
From which it follows that
\begin{align}
  \L &=   \log P(\vD,\H) - \KLD{Q}{P} \\
     &\le \log P(\vD,\H)
\end{align}
The probability of the model is 
\begin{align}
  P(\H, \vD) &=    \frac{P(\vD, \H) P(\H)}{P(\vD)}
             &\le  \frac{\L(Q)P(\H)}{P(\vD)}
\end{align}


\subsubsection{The model}
\begin{align}
  P(s_{m\omega}| \H) &= \sum_{c=1}^{N_c} \pi_{mc}\G(s_{m\omega};0,\beta_{\omega c})\\
  P(\beta_{\omega c} &= \GammaDistr(\beta_{mc} ; b^{(\beta)}, c^{(\beta)})\\
  P(\{\pi_{mc}\}_{c=1}^{N_c}|\H)  &= \Dirichlet\lr{ \{\pi_{mc}\}_{c=1}^{N_c} | c^{(\pi)}}
\end{align}
And mixture
\begin{align}
  P(A_{nm}|\H) &= \G(A_{nm}; 0,\alpha_m)\\
  P(\alpha_m| \H) &= \GammaDistr(\alpha_m| b^{(\alpha)},c^{(alpha)})
\end{align}
and gamma
\begin{align}
  P(\Lambda_{\omega},\H) = \GammaDistr(\Lambda_\omega;b^{(\Lambda)},c^{(\Lambda)} )
\end{align}

Simplify the distribution
\begin{align}
Q\lr{\vs, \vA, \pi, \beta, \alpha, \Lambda} = Q\lr{s_{\omega m}}Q\lr{A_{nm}}Q\lr{\pi}Q\lr{\beta}Q\lr{\alpha}Q\lr{\Lambda}
\end{align}

\begin{align}
  Q\lr{s_{nm}} = \G(s_{m \omega};\hat{s}_{m\omega}, \tilde{s}_{m\omega})\\
  Q\lr{A_{nm}} = \G(A_{mn};\hat{A}_{mn}, \tilde{A}_{mn})\\
  Q\lr{\beta_{mc}} = \Gamma(\beta_{mc};\hat{A}_{mn}, \tilde{A}_{mn})
\end{align}



%%% Local Variables: 
%%% mode: latex
%%% TeX-master: "tshorrock_thesis"
%%% End: 